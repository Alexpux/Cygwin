@node Signals
@chapter Signal Handling (@file{signal.h})

A @dfn{signal} is an event that interrupts the normal flow of control
in your program.  Your operating environment normally defines the full
set of signals available (see @file{sys/signal.h}), as well as the
default means of dealing with them---typically, either printing an
error message and aborting your program, or ignoring the signal.

All systems support at least the following signals:
@table @code
@item SIGABRT
Abnormal termination of a program; raised by the <<abort>> function.

@item SIGFPE
A domain error in arithmetic, such as overflow, or division by zero.

@item SIGILL
Attempt to execute as a function data that is not executable.

@item SIGINT
Interrupt; an interactive attention signal.

@item SIGSEGV
An attempt to access a memory location that is not available.

@item SIGTERM
A request that your program end execution.
@end table

Two functions are available for dealing with asynchronous
signals---one to allow your program to send signals to itself (this is
called @dfn{raising} a signal), and one to specify subroutines (called
@dfn{handlers} to handle particular signals that you anticipate may
occur---whether raised by your own program or the operating environment.

To support these functions, @file{signal.h} defines three macros:

@table @code
@item SIG_DFL
Used with the @code{signal} function in place of a pointer to a
handler subroutine, to select the operating environment's default
handling of a signal.

@item SIG_IGN
Used with the @code{signal} function in place of a pointer to a
handler, to ignore a particular signal.

@item SIG_ERR
Returned by the @code{signal} function in place of a pointer to a
handler, to indicate that your request to set up a handler could not
be honored for some reason.
@end table

@file{signal.h} also defines an integral type, @code{sig_atomic_t}.
This type is not used in any function declarations; it exists only to
allow your signal handlers to declare a static storage location where
they may store a signal value.  (Static storage is not otherwise
reliable from signal handlers.)

@menu
* psignal:: Print a signal message to standard error
* raise::   Send a signal
* signal::  Specify handler subroutine for a signal
@end menu

@page
@include signal/psignal.def

@page
@include signal/raise.def

@page
@include signal/signal.def
